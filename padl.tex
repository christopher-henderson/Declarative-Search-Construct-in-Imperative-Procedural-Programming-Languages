% This is samplepaper.tex, a sample chapter demonstrating the
% LLNCS macro package for Springer Computer Science proceedings;
% Version 2.20 of 2017/10/04
%
\documentclass[runningheads]{llncs}
%
\usepackage{graphicx}
\usepackage{syntax}


% Better inline directory listings
\usepackage{xcolor}
\definecolor{light-gray}{gray}{0.95}
% \newcommand{\code}[1]{\colorbox{light-gray}{\texttt{#1}}
\newcommand{\code}[1]{\texttt{#1}}
% Used for displaying a sample figure. If possible, figure files should
% be included in EPS format.
%
% If you use the hyperref package, please uncomment the following line
% to display URLs in blue roman font according to Springer's eBook style:
% \renewcommand\UrlFont{\color{blue}\rmfamily}

\begin{document}
%
\title{Declarative Search Construct in Imperative/Procedural Programming Languages}
%
%\titlerunning{Abbreviated paper title}
% If the paper title is too long for the running head, you can set
% an abbreviated paper title here
%
\author{Christopher Henderson\inst{1} \and Ajay Bansal\inst{1}}
% First names are abbreviated in the running head.
% If there are more than two authors, 'et al.' is used.
%
\institute{Arizona State University, Tempe AZ 85281, USA
\url{https://www.asu.edu}
}
%
\maketitle              % typeset the header of the contribution
%
\begin{abstract}
Graph theory is a critical component of computer science and software engineering, with algorithms concerning graph traversal and comprehension powering much of the complex problems in both industry and research. Engineers and researchers often have an accurate view of their target graph, however they struggle to implement a correct, and efficient, search over that graph. Even though modern imperative languages provide libraries for graph search, there are no built-in constructs for search with separation of concerns, that is separate the implementation of graph traversal from the user’s desired search result. In this paper, we propose a new programming language construct, the search statement, in order to facilitate rapid, correct, efficient, and intuitive development of graph-based solutions. Given a supra-root node, a procedure which determines the children of a given parent node, and optional definitions of the fail-fast acceptance or rejection of a solution, the search statement can conduct a search over any graph or network. Structurally, this statement is modeled after the common switch statement and is put into a largely imperative/procedural context to allow for declarative and intuitive development by most programmers. The Go programming language has been used as a foundation and proof-of-concept of the search statement. 

\keywords 
{
Backtracking \and
Graph Search \and
Imperative Programming \and
Declarative Programming
}

\end{abstract}
\section{EBNF Grammar}
This section provides the extended Backus-Naur form (EBNF) of the \code{search} construct. These grammars are only meaningful within the context of an extant grammar, the definition of which is outside the scope of this paper. As such, common definitions such as \code{statement-list} and \code{expr} have been elided and are to be taken as their common meanings.

\subsection{Idealic Grammar}
The following is the extended Backus-Naur form of the idealic candidate search construct.
\begin{grammar}
<search> ::= `search' <expr> [; <natural-number-expr>] `{' <search-block> `}'

<search-block> ::= `children' `:' <statement-list> [`accept' `:' <statement-list>][`reject' `:' <statement-list>]
\end{grammar}

\subsection{Implemented Grammar}
The following is the extended Backus-Naur form of the \code{search} construct as implemented in the Go programming language. It's defining difference is the admittance of a need to inform the compiler of the type of the initializing \code{expr} in the \code{search} siganture.

\begin{grammar}
<search> ::= `search' <expr>; <type-specifier> [; <natural-number-expr>] `{' <search-block> `}'

<search-block> ::= `children' `:' <statement-list> [`accept' `:' <statement-list>][`reject' `:' <statement-list>]
\end{grammar}

\end{document}
